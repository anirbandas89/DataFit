\documentclass[11pt,aps,prd,onecolumn,floatfix, tightenlines,shownopacs,showkeys,preprintnumbers,nofootinbib, longbibliography,notitlepage]{revtex4-1}
\pdfoutput=1
\usepackage[colorlinks=true,citecolor=red,linkcolor=blue,breaklinks=true]{hyperref}
\usepackage{amsmath,amssymb,mathtools,tabu}
\usepackage{epsfig, epstopdf}  
\usepackage{graphicx}   
\usepackage{commath}
\usepackage{slashed}             
\usepackage{url}
\usepackage{color}
\usepackage{multirow}
\usepackage{placeins}
\usepackage[dvipsnames]{xcolor}
\usepackage{comment}
%\usepackage[utf8]{inputenc}
\clubpenalty=1000
\widowpenalty=10000

\allowdisplaybreaks
\setlength{\bibsep}{0cm}
\bibpunct{[}{]}{,}{n}{}{,}

\newcommand{\ad}[1]{{\color{red}#1}}%red
\newcommand{\dth}{d^\mathrm{th}}
\newcommand{\lik}{\mathcal{L}}

\def\prd{Phys.Rev.D}         % Physical Review D
\def\apss{Astrophysics and Space Science}
\def\apj{ApJ}
\def\mnras{MNRAS}
\def\thesection{\Roman{section}}

\begin{document}
	\title{Notes on Model Fitting to Data}
	
	\author{Anirban Das}
	%\email{anirban@theory.tifr.res.in }                  
	%
	%\affiliation{Tata Institute of Fundamental Research,
	%             Homi Bhabha Road, Mumbai, 400005, India.}
	
	%\preprint{TIFR/TH/17-XX}
	\bigskip
	
	\date{\today}
	%=============================================================================
	
	\maketitle
	
	Suppose, we want to explain an observation of some spectrum using a theory model. The data is given by an array $d_i$. The corresponding theory predictions are $\dth_i$. This $\dth_i$ includes both signal and background, $\dth_i \equiv s_i + b_i$. Therefore the likelihood for this theory is
	\begin{equation}
	-2\ln\lik = \sum_i\dfrac{(d_i - \dth_i)^2}{\sigma_i^2}\,.
	\end{equation}
	Here $\sigma_i$ are the total systematic and statistical errors of the data. This is valid only if the data $d_i$ are independent. If this is not the case and they have a covariance matrix $\mathbf{V}$ then
	\begin{equation}
	-2\ln\lik = (\mathbf{d} - \mathbf{\dth})^T \mathbf{V}^{-1} (\mathbf{d} - \mathbf{\dth})\,.
	\end{equation}
	
	Now if the we define the \emph{null hypothesis} likelihood $\lik_0$, i.e., when the data is explained with the background only
	\begin{equation}
	-2\ln\lik_0 = \sum_i\dfrac{(d_i - b_i)^2}{\sigma_i^2}\,,
	\end{equation}
	then the \emph{test statistics} $TS$ is defined as
	\begin{equation}
	TS = -2\ln\left(\dfrac{\lik_0}{\lik}\right)\,.
	\end{equation}
	Basically, $TS$ is the difference between the $\chi^2$s of the fits with and without signal. It can be calculated for a range of points in the parameter space of the theory. The point having the greatest $TS$ is expected to be best description of the data.
	\bigskip
	
	\emph{We could just minimize $\chi^2$ to find the best-fit point. Why bother computing $TS$?}
	
	$TS$ is typically used for source detection in a skymap of, let's say gamma-ray. For example, Fermi collaboration uses this method to detect sources on the sky. In this case, the $TS$ formula is modified to the following
	\begin{equation}
	TS = -2\ln\left(\dfrac{\lik_\mathrm{max,0}}{\lik_\mathrm{max}}\right)
	\end{equation}
	where $\lik_\mathrm{max}$ is the maximum likelihood found by minimizing $\chi^2$ varying all parameters of the model. The $TS$ is then the relative measurement of fitting with and without a source at a particular position in the sky\footnote{\href{https://fermi.gsfc.nasa.gov/ssc/data/analysis/documentation/Cicerone/Cicerone\_Likelihood/Likelihood\_overview.html}{Fermi likelihood overview.}}.
	\bigskip
	
	\textbf{Small number of counts}
	
	If the count of particles in each data bin is small then the normal distribution cannot be used for likelihood calculation. The more appropriate function is the Poisson distribution,
	\begin{equation}
	P(n;\mu) = \dfrac{\mu^n \exp^{-\mu}}{n!}\,.
	\end{equation}
	Therefore
	\begin{equation}
	\lik = \prod_i \dfrac{(\dth_i)^{d_i} }{d_i!} \exp(-\dth_i) = \exp(-N_\mathrm{exp}) \prod_i \dfrac{(\dth_i)^{d_i}}{d_i!}\,,
	\end{equation}
	where $N_\mathrm{exp}$ is the total expected count in all bins\footnote{See \href{https://fermi.gsfc.nasa.gov/ssc/data/analysis/documentation/Cicerone/Cicerone\_Likelihood/Likelihood\_formula.html}{this} for more about binning issues.}.
\end{document}